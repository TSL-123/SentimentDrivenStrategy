
\documentclass[paper=a4, fontsize=11pt]{scrartcl} % A4 paper and 11pt font size

\usepackage[T1]{fontenc} % Use 8-bit encoding that has 256 glyphs
\usepackage{fourier} % Use the Adobe Utopia font for the document - comment this line to return to the LaTeX default
\usepackage[english]{babel} % English language/hyphenation
\usepackage{amsmath,amsfonts,amsthm} % Math packages

\usepackage{lipsum} % Used for inserting dummy 'Lorem ipsum' text into the template

\usepackage{sectsty} % Allows customizing section commands
\allsectionsfont{\centering \normalfont\scshape} % Make all sections centered, the default font and small caps

\usepackage{fancyhdr} % Custom headers and footers
\pagestyle{fancyplain} % Makes all pages in the document conform to the custom headers and footers
\fancyhead{} % No page header - if you want one, create it in the same way as the footers below
\fancyfoot[L]{} % Empty left footer
\fancyfoot[C]{} % Empty center footer
\fancyfoot[R]{\thepage} % Page numbering for right footer
\renewcommand{\headrulewidth}{0pt} % Remove header underlines
\renewcommand{\footrulewidth}{0pt} % Remove footer underlines
\setlength{\headheight}{11pt} % Customize the height of the header

\numberwithin{equation}{section} % Number equations within sections (i.e. 1.1, 1.2, 2.1, 2.2 instead of 1, 2, 3, 4)
\numberwithin{figure}{section} % Number figures within sections (i.e. 1.1, 1.2, 2.1, 2.2 instead of 1, 2, 3, 4)
\numberwithin{table}{section} % Number tables within sections (i.e. 1.1, 1.2, 2.1, 2.2 instead of 1, 2, 3, 4)

\setlength\parindent{0pt} % Removes all indentation from paragraphs - comment this line for an assignment with lots of text

%----------------------------------------------------------------------------------------
%	TITLE SECTION
%----------------------------------------------------------------------------------------

\newcommand{\horrule}[1]{\rule{\linewidth}{#1}} % Create horizontal rule command with 1 argument of height

\title{	
\normalfont \normalsize 
\textsc{\large Cornell University, Operations Research\&Information Engineering\\
ORIE4741 Learning with Big Messy Data} \\ [25pt] % Your university, school and/or department name(s)
\horrule{0.5pt} \\[0.4cm] % Thin top horizontal rule
\huge Sentiment Driven Strategy \\ % The assignment title
\normalsize Project Proposal\\
\horrule{1pt} \\[0.5cm] % Thick bottom horizontal rule
}

\author{Ziming Zeng(zz494)\\Tiansheng Liu(tl752)} % Your name

\date{}%omit date
%%\date{\normalsize\today} % Today's date or a custom date

\begin{document}

\maketitle % Print the title

%----------------------------------------------------------------------------------------
%	PROBLEM 1
%----------------------------------------------------------------------------------------

\section{Abstract}

%\lipsum[2] % Dummy text

We want to generate sentimental indicator from market news, test if sentiment among professionals correlates with price fluctuation in stock market and how they correlate. After that we want to fit the sentiment indicator in trading models, see if additional sentiment indicator makes model better. Finally we expect to modify and optimize the model to make it usable in trading.


\section{Questions We Care About}

Investigation about tweets' sentimental implication has revealed correlation between public opinions and stock prices, while major players in stock market are institutions and professionals. Will information and sentiment from professional channels be more effective in predicting stock trends? Can we put this into use in trading strategy?

\section{Dataset We Use}

We'll use financial news provided by Interactive Broker and corresponding market index as training set. Use streaming data as test set.

%----------------------------------------------------------------------------------------

\end{document}